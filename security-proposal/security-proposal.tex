\documentclass[conference]{IEEEtran}
% \IEEEoverridecommandlockouts
% The preceding line is only needed to identify funding in the first footnote. If that is unneeded, please comment it out.
\usepackage{cite}
\usepackage{amsmath,amssymb,amsfonts}
\usepackage{algorithmic}
\usepackage{graphicx}
\usepackage{textcomp}
\usepackage{xcolor}
\def\BibTeX{{\rm B\kern-.05em{\sc i\kern-.025em b}\kern-.08em
    T\kern-.1667em\lower.7ex\hbox{E}\kern-.125emX}}
% Definiciones propias
\usepackage{comandos}

\begin{document}

\title{
Authentication codes  \\
{\footnotesize Information security proposal for 2022 Wolfram summer school}
}

\author{
    \IEEEauthorblockN{Armando Benjamín Cruz Hinojosa}
    \IEEEauthorblockA{
        \textit{Universidad Nacional Autónoma de México}\\
        Mexico City, Mexico \\
        aleph\_g@ciencias.unam.mx
    }
}

\maketitle

\begin{abstract}
    This document is a model and instructions for \LaTeX.
    This and the IEEEtran.cls file define the components of your paper [title, text, heads, etc.]. *CRITICAL: Do Not Use Symbols, Special Characters, Footnotes,
    or Math in Paper Title or Abstract.
\end{abstract}

% \begin{IEEEkeywords}
%     component, formatting, style, styling, insert
% \end{IEEEkeywords}

%
% Introduction
%
\section{Introduction}
Authentication is a fundamental aspect of information security, it provides protection against impersonations and false messages. A strategy implemented in a comunication system to achieve such protection is called an \textit{authentication scheme}.

An authentication scheme is said to be secure if the if the chance that an impostor fools the system is so small that it does not represent a risk in the communication (comparable to channel noise).

A common scheme is to add a \textit{digital signature} at the end of each message. This signature is generated with asymetric cryptography: The message is compressed to a fixed size with a cryptographic hash function (like SHA-256), this hash is then encrypted with the emisor's private key. In order to verify the message, the reciever compares the message's hash with the decripted signature if they differ the message is forged.

Without the private key the best the enemy can do is try different keys until the message is accepted. This takes $2^{n-1}$ tries on average ($n$ is the size of the key). HTTPS specifies a maximum key size of 2048 bits and RSA encryption. A brute ofrce atack would take $2^{2047} \approx 10^{616}$ tries, impossible task for current computers.

But if an eficient solution to the factorization into prime factors is used to break the key (shors algorithm) % \cite{ariculoShore}
Or if the impostor's computational power drastically increases, the digital signature is useless for authentication protection.

With this in mind cite{SS} proposes a different kind of authentication scheme, one that remains secure regardless of the impostor's computational power. Three parts interact in the comunication system, \A\ an information source that wishes to communicate an element of the set of all possible source states $\s$, \B\ a reciever and \E\ the impostor.

\A\ will transform the source states into \textit{cyphered messages} (sequences compatible with the communication channel) by means of an \textit{encoding rule}. An encoding rule $r\in\r$ is a one-to-one mapping from $\s$ to $\m$. \B\ accepts a message $m$ as authentic if $m\in r[\s]$ and the original state is the pre-image $r^{-1}(m)$. This schema is represented in Figure~\ref{figModelo3Participantes}.

\begin{figure}
    \centering
    \resizebox{4.5px}{!}{
        \begin{tikzpicture}[trim left=(channel), trim right=(channel), auto, node distance=2cm,>=latex, scale=0.50]
            \node [smallblock](channel){};
            \node [block, left of=channel, node distance=3cm, label=below:Codificador](encoder){$ \r $};
            \node [block, left of=encoder, node distance=3cm, label=below:{\parbox{6em}{\centering Fuente de \\ Información}}](infosource){\A};
            \node (decoder)[block, right of=channel, node distance=3cm, label=below:Decodificador]{};
            \node (destination)[block, right of=decoder, node distance=3cm, label=below:Destino]{\B};
            \node (enemy)[block, below of=channel, label=below:Enemigo]{\E};

            \draw [->] (infosource) -- node[name=u]{\textit{estado}} (encoder);
            \draw [->] (encoder) -- node {\parbox{2cm}{\centering \textit{mensaje codificado}}} (channel);
            \draw [->] (channel) -- node {\parbox{2cm}{\centering \textit{mensaje recibido}}} (decoder);
            \draw [->] (decoder) -- node [name=y]{\textit{estado}}(destination);
            \draw [->] (enemy) -- node[name=enemychannel]{} (channel);
            \node [left of=enemychannel, node distance=7mm]{\parbox{2cm}{\centering \textit{mensaje falso}}};
        \end{tikzpicture}
    }
    \caption{Sistema de comunicación del modelo con tres participantes.}
    \label{figModelo3Participantes}
\end{figure}

An instance of this schema is a 5-tuple $\aprottuple$, where $\ped{S}$ denotes a stochastic process over the set of all source states ie. the \textit{information source}.

What we know of authentication codes. theoretical lower bound, the inconditional suplat probability.

%
% Objective
%
\section{Objective}
I want to statistically verify the theorem. What is the distribution.

Also I want to calculate the values.

%
% Plan
%
\section{Plan}
Before you begin to format your paper, first write and save the content as a
separate text file. Complete all content and organizational editing before
formatting.
proofreading, spelling and grammar.

\subsection{Information source}\label{AA}
Define abbreviations and acronyms the first time they are used in the text,
even after they have been defined in the abstract. Abbreviations such as
IEEE, SI, MKS, CGS, ac, dc, and rms do not have to be defined. Do not use
abbreviations in the title or heads unless they are unavoidable.

\subsection{Balanced t-designs}

\subsection{Strategy simulation}

Be sure that the
symbols in your equation have been defined before or immediately following

% \begin{figure}[htbp]
%     \centerline{\includegraphics{fig1.png}}
%     \caption{Example of a figure caption.}
%     \label{fig}
% \end{figure}

Figure Labels: Use 8 point Times New Roman for Figure labels. Use words
rather than symbols or abbreviations when writing Figure axis labels to
avoid confusing the reader. As an example, write the quantity
``Magnetization'', or ``Magnetization, M'', not just ``M''. If including
units in the label, present them within parentheses. Do not label axes only
with units. In the example, write ``Magnetization (A/m)'' or ``Magnetization
\{A[m(1)]\}'', not just ``A/m''. Do not label axes with a ratio of
quantities and units. For example, write ``Temperature (K)'', not.

%
% Familiarity with the problem
%
\section{Familiarity with the problem}
I have been working with the codes for one year for my dissertation, I know the theorem and the deduction.

%
% Bibliografía
%
\begin{thebibliography}{00}
    \bibitem{b1} G. Eason, B. Noble, and I. N. Sneddon, ``On certain integrals of Lipschitz-Hankel type involving products of Bessel functions,'' Phil. Trans. Roy. Soc. London, vol. A247, pp. 529--551, April 1955.
    \bibitem{b2} J. Clerk Maxwell, A Treatise on Electricity and Magnetism, 3rd ed., vol. 2. Oxford: Clarendon, 1892, pp.68--73.
    \bibitem{b3} I. S. Jacobs and C. P. Bean, ``Fine particles, thin films and exchange anisotropy,'' in Magnetism, vol. III, G. T. Rado and H. Suhl, Eds. New York: Academic, 1963, pp. 271--350.
    \bibitem{b4} K. Elissa, ``Title of paper if known,'' unpublished.
    \bibitem{b5} R. Nicole, ``Title of paper with only first word capitalized,'' J. Name Stand. Abbrev., in press.
    \bibitem{b6} Y. Yorozu, M. Hirano, K. Oka, and Y. Tagawa, ``Electron spectroscopy studies on magneto-optical media and plastic substrate interface,'' IEEE Transl. J. Magn. Japan, vol. 2, pp. 740--741, August 1987 [Digests 9th Annual Conf. Magnetics Japan, p. 301, 1982].
    \bibitem{b7} M. Young, The Technical Writer's Handbook. Mill Valley, CA: University Science, 1989.
\end{thebibliography}

\end{document}
