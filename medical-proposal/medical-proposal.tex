\documentclass[conference]{IEEEtran}
% \IEEEoverridecommandlockouts
% The preceding line is only needed to identify funding in the first footnote. If that is unneeded, please comment it out.
\usepackage{cite}
\usepackage{amsmath,amssymb,amsfonts}
\usepackage{algorithmic}
\usepackage{graphicx}
\usepackage{textcomp}
\usepackage{xcolor}
\def\BibTeX{{\rm B\kern-.05em{\sc i\kern-.025em b}\kern-.08em
    T\kern-.1667em\lower.7ex\hbox{E}\kern-.125emX}}
% Definiciones propias
\usepackage{comandos}

\begin{document}

\title{
Tractography ML  \\
{\footnotesize Medical statistics proposal for 2022 Wolfram summer school}
}

\author{
    \IEEEauthorblockN{Armando Benjamín Cruz Hinojosa}
    \IEEEauthorblockA{
        \textit{Universidad Nacional Autónoma de México}\\
        Mexico City, Mexico \\
        aleph\_g@ciencias.unam.mx
    }
}

\maketitle

\begin{abstract}
    Current authentication schemes are based on asymmetric cryptography protocols like RSA, with the arrival of cuantum computing and fast factorization algorithms, security is endangered. A-schemes presents an alternative that remains secure regardless of computational power, but there are no tools to measure how secure they are. This proposal intends to create such tool.
\end{abstract}

% \begin{IEEEkeywords}
%     component, formatting, style, styling, insert
% \end{IEEEkeywords}

%
% Introduction
%
\section{Introduction}
Talk about difussion MRI

Talk about the contest

Talk about a solution and previous work

%
% Objective
%
\section{Objective}
Implement and use wolfram ML classifiers and compare the solutions

%
% Implementation
%
\section{Implementation}
Use tractosplit


% \begin{figure}[htbp]
%     \centerline{\includegraphics{fig1.png}}
%     \caption{Example of a figure caption.}
%     \label{fig}
% \end{figure}

%
% Familiarity with the problem
%
\section{Familiarity with the problem}
I have been working with tractograms.

%
% Bibliografía
%
\begin{thebibliography}{00}
    \bibitem{b1} G. Eason, B. Noble, and I. N. Sneddon, ``On certain integrals of Lipschitz-Hankel type involving products of Bessel functions,'' Phil. Trans. Roy. Soc. London, vol. A247, pp. 529--551, April 1955.
    \bibitem{b2} J. Clerk Maxwell, A Treatise on Electricity and Magnetism, 3rd ed., vol. 2. Oxford: Clarendon, 1892, pp.68--73.
    \bibitem{b3} I. S. Jacobs and C. P. Bean, ``Fine particles, thin films and exchange anisotropy,'' in Magnetism, vol. III, G. T. Rado and H. Suhl, Eds. New York: Academic, 1963, pp. 271--350.
    \bibitem{b4} K. Elissa, ``Title of paper if known,'' unpublished.
    \bibitem{b5} R. Nicole, ``Title of paper with only first word capitalized,'' J. Name Stand. Abbrev., in press.
    \bibitem{b6} Y. Yorozu, M. Hirano, K. Oka, and Y. Tagawa, ``Electron spectroscopy studies on magneto-optical media and plastic substrate interface,'' IEEE Transl. J. Magn. Japan, vol. 2, pp. 740--741, August 1987 [Digests 9th Annual Conf. Magnetics Japan, p. 301, 1982].
    \bibitem{b7} M. Young, The Technical Writer's Handbook. Mill Valley, CA: University Science, 1989.
\end{thebibliography}

\end{document}
